\documentclass[11pt]{scrartcl}
\usepackage[sexy]{evan}





\begin{document}
\title{Inequalities}
\maketitle

Must-do thing in general:
\begin{itemize}
    \item Affine transformation on the variables, and smoothing.
    \item Check equality case to decide how to bound, i.e. which inequality to use, or how to complete the square then $X^2 \ge 0$
    \item Handle odd and even cases separately, especially when pairing is possible.
    \item When there is $x_ix_{i+1}$, consider AM-GM/Cauchy, or $(\sum x_i)^2$, or take partial sum, see: IMO SL 2015 A3.
    \item Change the expression to as symmetric as possible (easier to handle).
\end{itemize}

\tableofcontents




\clearpage


\begin{Problem}[IMO SL 2016 A2: \url{https://artofproblemsolving.com/community/c6h1480705p8639281}]
Find the smallest constant $C > 0$ for which the following statement holds: among any five positive real numbers $a_1,a_2,a_3,a_4,a_5$ (not necessarily distinct), one can always choose distinct subscripts $i,j,k,l$ such that
\[ \left| \frac{a_i}{a_j} - \frac {a_k}{a_l} \right| \le C. \]
\end{Problem}




\begin{Problem}[IMO SL 2016 A3: \url{https://artofproblemsolving.com/community/c6h1480716p8639312}]
Find all positive integers $n$ such that the following statement holds: Suppose real numbers $a_1$, $a_2$, $\dots$, $a_n$, $b_1$, $b_2$, $\dots$, $b_n$ satisfy $|a_k|+|b_k|=1$ for all $k=1,\dots,n$. Then there exists $\varepsilon_1$, $\varepsilon_2$, $\dots$, $\varepsilon_n$, each of which is either $-1$ or $1$, such that
\[ \left| \sum_{i=1}^n \varepsilon_i a_i \right| + \left| \sum_{i=1}^n \varepsilon_i b_i \right| \le 1. \]
\end{Problem}




\begin{Problem}[IMO SL 2016 A8: \url{https://artofproblemsolving.com/community/c6h1480693p8639259}]
Find the largest real constant $a$ such that for all $n \geq 1$ and for all real numbers $x_0, x_1, ... , x_n$ satisfying $0 = x_0 < x_1 < x_2 < \cdots < x_n$ we have
\[\frac{1}{x_1-x_0} + \frac{1}{x_2-x_1} + \dots + \frac{1}{x_n-x_{n-1}} \geq a \left( \frac{2}{x_1} + \frac{3}{x_2} + \dots + \frac{n+1}{x_n} \right)\]
\end{Problem}



\begin{Problem}[IMO SL 2020 A1: \url{https://artofproblemsolving.com/community/c6h2625913p22698446}]
Version 1. Let $n$ be a positive integer, and set $N=2^{n}$. Determine the smallest real number $a_{n}$ such that, for all real $x$,
\[
\sqrt[N]{\frac{x^{2 N}+1}{2}} \leqslant a_{n}(x-1)^{2}+x .
\]Version 2. For every positive integer $N$, determine the smallest real number $b_{N}$ such that, for all real $x$,
\[
\sqrt[N]{\frac{x^{2 N}+1}{2}} \leqslant b_{N}(x-1)^{2}+x .
\]
\end{Problem}




\begin{Problem}[IMO SL 2022 A4: \url{https://artofproblemsolving.com/community/c6h3107365p28104481}]
Let $n \geqslant 3$ be an integer, and let $x_1,x_2,\ldots,x_n$ be real numbers in the interval $[0,1]$. Let $s=x_1+x_2+\ldots+x_n$, and assume that $s \geqslant 3$. Prove that there exist integers $i$ and $j$ with $1 \leqslant i<j \leqslant n$ such that
\[2^{j-i}x_ix_j>2^{s-3}.\]
\end{Problem}

\begin{Problem}[IMO SL 2020 A3: \url{https://artofproblemsolving.com/community/c6h2625839p22697887}]
Suppose that $a,b,c,d$ are positive real numbers satisfying $(a+c)(b+d)=ac+bd$. Find the smallest possible value of
$$\frac{a}{b}+\frac{b}{c}+\frac{c}{d}+\frac{d}{a}.$$
\end{Problem}



\begin{Problem}[IMO SL 2019 A4: \url{https://artofproblemsolving.com/community/c6h2279021p17829013}]
Let $n\geqslant 2$ be a positive integer and $a_1,a_2, \ldots ,a_n$ be real numbers such that\[a_1+a_2+\dots+a_n=0.\]Define the set $A$ by
\[A=\left\{(i, j)\,|\,1 \leqslant i<j \leqslant n,\left|a_{i}-a_{j}\right| \geqslant 1\right\}\]Prove that, if $A$ is not empty, then
\[\sum_{(i, j) \in A} a_{i} a_{j}<0.\]
\end{Problem}

\begin{Problem}[IMO SL 2017 A4: \url{https://artofproblemsolving.com/community/c6h1671279p10632313}]
A sequence of real numbers $a_1,a_2,\ldots$ satisfies the relation
$$a_n=-\max_{i+j=n}(a_i+a_j)\qquad\text{for all}\quad n>2017.$$Prove that the sequence is bounded, i.e., there is a constant $M$ such that $|a_n|\leq M$ for all positive integers $n$.
\end{Problem}

\begin{Problem}[IMO SL 2017 A5: \url{https://artofproblemsolving.com/community/c6h1671280p10632315}]
An integer $n \geq 3$ is given. We call an $n$-tuple of real numbers $(x_1, x_2, \dots, x_n)$ Shiny if for each permutation $y_1, y_2, \dots, y_n$ of these numbers, we have
$$\sum \limits_{i=1}^{n-1} y_i y_{i+1} = y_1y_2 + y_2y_3 + y_3y_4 + \cdots + y_{n-1}y_n \geq -1.$$Find the largest constant $K = K(n)$ such that
$$\sum \limits_{1 \leq i < j \leq n} x_i x_j \geq K$$holds for every Shiny $n$-tuple $(x_1, x_2, \dots, x_n)$.
\end{Problem}




\begin{Problem}[IMO SL 2018 A7: \url{https://artofproblemsolving.com/community/c6h1876747p12752777}]
Find the maximal value of
\[S = \sqrt[3]{\frac{a}{b+7}} + \sqrt[3]{\frac{b}{c+7}} + \sqrt[3]{\frac{c}{d+7}} + \sqrt[3]{\frac{d}{a+7}},\]where $a$, $b$, $c$, $d$ are nonnegative real numbers which satisfy $a+b+c+d = 100$.
\end{Problem}


\begin{Problem}[IMO SL 2019 A3: \url{https://artofproblemsolving.com/community/c6h2279061p17829510}]
Let $n \geqslant 3$ be a positive integer and let $\left(a_{1}, a_{2}, \ldots, a_{n}\right)$ be a strictly increasing sequence of $n$ positive real numbers with sum equal to 2. Let $X$ be a subset of $\{1,2, \ldots, n\}$ such that the value of
\[
\left|1-\sum_{i \in X} a_{i}\right|
\]is minimised. Prove that there exists a strictly increasing sequence of $n$ positive real numbers $\left(b_{1}, b_{2}, \ldots, b_{n}\right)$ with sum equal to 2 such that
\[
\sum_{i \in X} b_{i}=1.
\]
\end{Problem}


\begin{Problem}[IMO SL 2019 A2: \url{https://artofproblemsolving.com/community/c6h2279014p17828913}]
Let $u_1, u_2, \dots, u_{2019}$ be real numbers satisfying\[u_{1}+u_{2}+\cdots+u_{2019}=0 \quad \text { and } \quad u_{1}^{2}+u_{2}^{2}+\cdots+u_{2019}^{2}=1.\]Let $a=\min \left(u_{1}, u_{2}, \ldots, u_{2019}\right)$ and $b=\max \left(u_{1}, u_{2}, \ldots, u_{2019}\right)$. Prove that
\[
a b \leqslant-\frac{1}{2019}.
\]
\end{Problem}


\begin{Problem}[IMO SL 2022 C1: \url{https://artofproblemsolving.com/community/c6h3107334p28104276}]
A $\pm 1$-sequence is a sequence of $2022$ numbers $a_1, \ldots, a_{2022},$ each equal to either $+1$ or $-1$. Determine the largest $C$ so that, for any $\pm 1$-sequence, there exists an integer $k$ and indices $1 \le t_1 < \ldots < t_k \le 2022$ so that $t_{i+1} - t_i \le 2$ for all $i$, and$$\left| \sum_{i = 1}^{k} a_{t_i} \right| \ge C.$$
\end{Problem}


\begin{Problem}[IMO SL 2021 A1: \url{https://artofproblemsolving.com/community/c6h2882570p25627665}]
Let $n$ be a positive integer. Given $A$ is a subset of $\{0,1,...,5^n\}$ with $4n+2$ elements. Prove that there exist three elements $a<b<c$ from $A$ such that $c+2a>3b$.
\end{Problem}







\begin{Problem}[IMO SL 2021 A5: \url{https://artofproblemsolving.com/community/c6h2882545p25627517}]
Let $n\geq 2$ be an integer and let $a_1, a_2, \ldots, a_n$ be positive real numbers with sum $1$. Prove that$$\sum_{k=1}^n \frac{a_k}{1-a_k}(a_1+a_2+\cdots+a_{k-1})^2 < \frac{1}{3}.$$
\end{Problem}


\begin{Problem}[IMO SL 2021 A7: \url{https://artofproblemsolving.com/community/c6h2882578p25627708}]
Let $n\geqslant 1$ be an integer, and let $x_0,x_1,\ldots,x_{n+1}$ be $n+2$ non-negative real numbers that satisfy $x_ix_{i+1}-x_{i-1}^2\geqslant 1$ for all $i=1,2,\ldots,n.$ Show that\[x_0+x_1+\cdots+x_n+x_{n+1}>\bigg(\frac{2n}{3}\bigg)^{3/2}.\]
\end{Problem}









\begin{Problem}[IMO SL 2023 A1: \url{https://artofproblemsolving.com/community/c6h3359728p31218381}]
Professor Oak is feeding his $100$ Pokémon. Each Pokémon has a bowl whose capacity is a positive real number of kilograms. These capacities are known to Professor Oak. The total capacity of all the bowls is $100$ kilograms. Professor Oak distributes $100$ kilograms of food in such a way that each Pokémon receives a non-negative integer number of kilograms of food (which may be larger than the capacity of the bowl). The dissatisfaction level of a Pokémon who received $N$ kilograms of food and whose bowl has a capacity of $C$ kilograms is equal to $\lvert N-C\rvert$.

Find the smallest real number $D$ such that, regardless of the capacities of the bowls, Professor Oak can distribute food in a way that the sum of the dissatisfaction levels over all the $100$ Pokémon is at most $D$.
\end{Problem}








\begin{Problem}[IMO SL 2023 A5]
Let $a_1,a_2,\dots,a_{2023}$ be positive integers such that
$a_1,a_2,\dots,a_{2023}$ is a permutation of $1,2,\dots,2023$, and
$|a_1-a_2|,|a_2-a_3|,\dots,|a_{2022}-a_{2023}|$ is a permutation of $1,2,\dots,2022$.
Prove that $\max(a_1,a_{2023})\ge 507$.
\end{Problem}








\begin{Problem}[IMO SL 2011 A7: \url{https://artofproblemsolving.com/community/c6h488538p2737646}]
Let $a,b$ and $c$ be positive real numbers satisfying $\min(a+b,b+c,c+a) > \sqrt{2}$ and $a^2+b^2+c^2=3.$ Prove that

\[\frac{a}{(b+c-a)^2} + \frac{b}{(c+a-b)^2} + \frac{c}{(a+b-c)^2} \geq \frac{3}{(abc)^2}.\]
\end{Problem}

\begin{Problem}[IMO SL 2011 A1: \url{https://artofproblemsolving.com/community/c6h418793p2363530}]
Given any set $A = \{a_1, a_2, a_3, a_4\}$ of four distinct positive integers, we denote the sum $a_1 +a_2 +a_3 +a_4$ by $s_A$. Let $n_A$ denote the number of pairs $(i, j)$ with $1 \leq  i < j \leq 4$ for which $a_i +a_j$ divides $s_A$. Find all sets $A$ of four distinct positive integers which achieve the largest possible value of $n_A$.
\end{Problem}




\begin{Problem}[IMO SL 2010 A7: \url{https://artofproblemsolving.com/community/c6h356197p1936918}]
Let $a_1, a_2, a_3, \ldots$ be a sequence of positive real numbers, and $s$ be a positive integer, such that
\[a_n = \max \{ a_k + a_{n-k} \mid 1 \leq k \leq n-1 \} \ \textrm{ for all } \ n > s.\]
Prove there exist positive integers $\ell \leq s$ and $N$, such that
\[a_n = a_{\ell} + a_{n - \ell} \ \textrm{ for all } \ n \geq N.\]
\end{Problem}


\begin{Problem}[IMO SL 2010 A8: \url{https://artofproblemsolving.com/community/c6h418684p2362291}]
Given six positive numbers $a,b,c,d,e,f$ such that $a < b < c < d < e < f.$ Let $a+c+e=S$ and $b+d+f=T.$ Prove that
\[2ST > \sqrt{3(S+T)\left(S(bd + bf + df) + T(ac + ae + ce) \right)}.\]
\end{Problem}




\clearpage

\section{Just algebra}
\begin{itemize}
    \item Application of classical inequalities
    \item Tangent line trick
\end{itemize}

\begin{Problem}[IMO SL 2017 A1: \url{https://artofproblemsolving.com/community/c6h1671272p10632289}]
Let $a_1,a_2,\ldots a_n,k$, and $M$ be positive integers such that
$$\frac{1}{a_1}+\frac{1}{a_2}+\cdots+\frac{1}{a_n}=k\quad\text{and}\quad a_1a_2\cdots a_n=M.$$If $M>1$, prove that the polynomial
$$P(x)=M(x+1)^k-(x+a_1)(x+a_2)\cdots (x+a_n)$$has no positive roots.
\end{Problem}


\begin{Problem}[IMO SL 2020 A4: \url{https://artofproblemsolving.com/community/c6h2278647p17821569}]
The real numbers $a, b, c, d$ are such that $a\geq b\geq c\geq d>0$ and $a+b+c+d=1$. Prove that
\[(a+2b+3c+4d)a^ab^bc^cd^d<1\]
\end{Problem}
\begin{lesson}
Expand and compare terms
\end{lesson}


\subsection{Convexity}
\begin{Problem}[IMO SL 2021 A4: \url{https://artofproblemsolving.com/community/c6h2625850p22697952}]
Show that the inequality\[\sum_{i=1}^n \sum_{j=1}^n \sqrt{|x_i-x_j|}\leqslant \sum_{i=1}^n \sum_{j=1}^n \sqrt{|x_i+x_j|}\]holds for all real numbers $x_1,\ldots x_n.$
\end{Problem}




\section{Local bounding, then take union bound}

\begin{Problem}[IMO SL 2010 A3: \url{https://artofproblemsolving.com/community/c6h418680p2362280}]
Let $x_1, \ldots , x_{100}$ be nonnegative real numbers such that $x_i + x_{i+1} + x_{i+2} \leq 1$ for all $i = 1, \ldots , 100$ (we put $x_{101 } = x_1, x_{102} = x_2).$ Find the maximal possible value of the sum $S = \sum^{100}_{i=1} x_i x_{i+2}.$
\end{Problem}



\section{Match coefficients}
\begin{Problem}[IMO SL 2010 A2: \url{https://artofproblemsolving.com/community/c6h418679p2362276}]
Let the real numbers $a,b,c,d$ satisfy the relations $a+b+c+d=6$ and $a^2+b^2+c^2+d^2=12.$ Prove that
\[36 \leq 4 \left(a^3+b^3+c^3+d^3\right) - \left(a^4+b^4+c^4+d^4 \right) \leq 48.\]
\end{Problem}
\begin{lesson}
Since $a,b,c,d \in \RR$ (we will need to do SOS etc), I want to create $\cycsum (a-\lambda)^2 = 0$, but after careful thought, 0 is impossible. I created $\cycsum (a-1)^2$. And noting that the expressions desired appear in $(a-1)^4$, I can assure that this will work.
\end{lesson}
\[4 = \cycsum a^2 - 2\cycsum a + \cycsum 1 = \cycsum (a-1)^2\]
Note that $4 = \binom{4}{1}$, so consider
\[\cycsum (a-1)^4 = \cycsum a^4 - 4 \cycsum a^3 + 6 \cycsum a^2 - 4 \cycsum a + \cycsum 1\]
So
\[4\cycsum a^3 - \cycsum a^4 = 6(12) - 4(6) + 4 - \cycsum (a-1)^4 = 52 - \cycsum (a-1)^4\]
Then the original inequality is equivalent to
\[4 \le \cycsum (a-1)^4 \le 16\]
Note that by expansion, since $(a-1)^2 \ge 0$,
\[16 = 4^2 = \PR{\cycsum (a-1)^2}^2 \ge \cycsum (a-1)^4\]
and for the other inequality, by Cauchy-Schwarz,
\[\PR{\cycsum 1} \PR{\cycsum (a-1)^4} \ge \PR{\cycsum (a-1)^2}^2 = 16 \iff \cycsum (a-1)^4 \ge 4\]
This finishes the proof.



\section{Sequences - Algebra}

\begin{itemize}
    \item Rewrite the problem condition into nicer look
    \item Force telescoping
    \item Induction
\end{itemize}

\begin{Problem}[IMO SL 2015 A1: \url{https://artofproblemsolving.com/community/c6h1268809p6621766}]
Suppose that a sequence $a_1,a_2,\ldots$ of positive real numbers satisfies\[a_{k+1}\geq\frac{ka_k}{a_k^2+(k-1)}\]for every positive integer $k$. Prove that $a_1+a_2+\ldots+a_n\geq n$ for every $n\geq2$.
\end{Problem}


\begin{Problem}[IMO SL 2022 A1: \url{https://artofproblemsolving.com/community/c6h3107323p28104243}]
Let $(a_n)_{n\geq 1}$ be a sequence of positive real numbers with the property that
$$(a_{n+1})^2 + a_na_{n+2} \leq a_n + a_{n+2}$$for all positive integers $n$. Show that $a_{2022}\leq 1$.
\end{Problem}





\subsection{Induction-type, Local, Do cases}
\begin{Problem}[IMO SL 2010 A4: \url{https://artofproblemsolving.com/community/c6h418681p2362283}]
A sequence $x_1, x_2, \ldots$ is defined by $x_1 = 1$ and $x_{2k}=-x_k, x_{2k-1} = (-1)^{k+1}x_k$ for all $k \geq 1.$ Prove that $\forall n \geq 1$, $x_1 + x_2 + \ldots + x_n \geq 0.$
\end{Problem}
\begin{lesson}
When there's +1 and -1, consider modulo 2.
\end{lesson}


\begin{Problem}[IMO SL 2023 A3: \url{https://artofproblemsolving.com/community/c6h3107339p28104298}]
Let $x_1,x_2,\dots,x_{2023}$ be pairwise different positive real numbers such that
\[a_n=\sqrt{(x_1+x_2+\dots+x_n)\left(\frac{1}{x_1}+\frac{1}{x_2}+\dots+\frac{1}{x_n}\right)}\]is an integer for every $n=1,2,\dots,2023.$ Prove that $a_{2023} \geqslant 3034.$
\end{Problem}




\section{Sequences - Combinatorial Idea (Extremal etc.)}

\begin{Problem}[IMO 2014 A1: \url{https://artofproblemsolving.com/community/c6h596930p3542095}]
Let $a_0 < a_1 < a_2 < \dots$ be an infinite sequence of positive integers. Prove that there exists a unique integer $n\geq 1$ such that
\[a_n < \frac{a_0+a_1+a_2+\cdots+a_n}{n} \leq a_{n+1}.\]
\end{Problem}

\begin{motivation}
Note that the right inequality is loose, because $a_{n+1}$ is unrelated, except that $a_{n+1} > a_n$. Hence the focus can be put on $a_n$ "solely". If that's the case, we shall pick the first $n$ such that the left inequality holds.
\end{motivation}

Let $W(n) = (a_0 + ... + a_n) - na_n$. Then
\[W(n+1) - W(n) = a_{n+1} - (n+1)a_{n+1} + na_n = n(a_n - a_{n+1}) < 0\]
So $W(n)$ is strictly decreasing, i.e. the left inequality must fail for all sufficiently large $n$.\\

Pick $n$ as the largest $n$ such that $W(n) > 0$. By definition, $W(n+1) \le 0$, i.e.
\[(n+1)a_{n+1} \ge a_0 + ... + a_{n+1}\]
\[\iff a_0 + ... + a_n \le na_{n+1}\]
which is exactly the right inequality.


\begin{Problem}[IMO SL 2021 A3: \url{https://artofproblemsolving.com/community/c6h2882539p25627503}]
For each integer $n\ge 1,$ compute the smallest possible value of\[\sum_{k=1}^{n}\left\lfloor\frac{a_k}{k}\right\rfloor\]over all permutations $(a_1,\dots,a_n)$ of $\{1,\dots,n\}.$
\end{Problem}






\section{Sequences - Consider partial sum, block of sum, cumulative sum}
\begin{Problem}[IMO SL 2014 A3: \url{https://artofproblemsolving.com/community/c6h1113180p5083537}]
For a sequence $x_1,x_2,\ldots,x_n$ of real numbers, we define its $\textit{price}$ as\[\max_{1\le i\le n}|x_1+\cdots +x_i|.\]Given $n$ real numbers, Dave and George want to arrange them into a sequence with a low price. Diligent Dave checks all possible ways and finds the minimum possible price $D$. Greedy George, on the other hand, chooses $x_1$ such that $|x_1 |$ is as small as possible; among the remaining numbers, he chooses $x_2$ such that $|x_1 + x_2 |$ is as small as possible, and so on. Thus, in the $i$-th step he chooses $x_i$ among the remaining numbers so as to minimise the value of $|x_1 + x_2 + \cdots  x_i |$. In each step, if several numbers provide the same value, George chooses one at random. Finally he gets a sequence with price $G$.

Find the least possible constant $c$ such that for every positive integer $n$, for every collection of $n$ real numbers, and for every possible sequence that George might obtain, the resulting values satisfy the inequality $G\le cD$.
\end{Problem}



\begin{Problem}[IMO SL 2015 A3: \url{https://artofproblemsolving.com/community/c6h1268841p6622146}]
Let $n$ be a fixed positive integer. Find the maximum possible value of\[ \sum_{1 \le r < s \le 2n} (s-r-n)x_rx_s, \]where $-1 \le x_i \le 1$ for all $i = 1, \cdots , 2n$.
\end{Problem}
Solution: \url{https://artofproblemsolving.com/community/c6h1268841p31528092}





\begin{Problem}[IMO SL 2013 A4: \url{https://artofproblemsolving.com/community/c6h597122p3543362}]
Let $n$ be a positive integer, and consider a sequence $a_1 , a_2 , \dotsc , a_n $ of positive integers. Extend it periodically to an infinite sequence $a_1 , a_2 , \dotsc $ by defining $a_{n+i} = a_i $ for all $i \ge 1$. If\[a_1 \le a_2 \le \dots \le a_n \le a_1 +n  \]and\[a_{a_i } \le n+i-1 \quad\text{for}\quad i=1,2,\dotsc, n, \]prove that\[a_1 + \dots +a_n \le n^2. \]
\end{Problem}


\begin{motivation}
Since I want to tighten the bound so I consider splitting into block of equal value. Say we have
\[a_1 = a_{b_1} < a_{b_1 + b_2} < a_{b_1 + b_2 + b_3} < ... < a_{b_1 + ... + b_k} = a_n\]
so there're $k$ values, and I can use the smallest $i$. However, in order to bound $a_1 + ... + a_n$, I will eventually define/split $a_{a_1}, a_{a_2}, ..., a_{a_n}$ in the sequence. And there are a lot of things that I can't tell, so may end up to be a very blind bound. So maybe a crude bound should be enough.
\end{motivation}

Following the idea, we need to look at $a_{a_1}, a_{a_2}, ..., a_{a_n}$.
\begin{align*}
    a_1 + ... + a_{a_1} &\le n \cdot a_1\\
    a_{a_1 + 1} + ... + a_{a_2} &\le (n+1)(a_2 - a_1)\\
    a_{a_2 + 1} + ... + a_{a_3} &\le (n+2)(a_3-a_2)\\
    &\vdots\\
    a_{a_{n-1} + 1} + ... + a_{a_n} &\le (2n-1)(a_n - a_{n-1})\\
    a_{a_n + 1} + ... + a_n &\le (2n-1)(n-a_n)
\end{align*}
Letting $S = a_1 + ... + a_n$, summing all the inequalities,
\[S \le -(a_1 + ... + a_{n-1}) + n(2n-1)\]
\[\iff 2S \le n(2n-1) + a_n\]
\begin{itemize}
    \item If $a_n \ge n$, then $2S \le n(2n-1)+n = 2n^2 \implies S \le n^2$.
    \item If $a_n < n$, then $S \le n \cdots a_n < n^2$.
\end{itemize}









\section{Smoothing}

\begin{Problem}[IMO SL 2016 A1: \url{https://artofproblemsolving.com/community/c6h1480690p8639254}]
Let $a$, $b$, $c$ be positive real numbers such that $\min(ab,bc,ca) \ge 1$. Prove that$$\sqrt[3]{(a^2+1)(b^2+1)(c^2+1)} \le \left(\frac{a+b+c}{3}\right)^2 + 1.$$
\end{Problem}




\end{document}














