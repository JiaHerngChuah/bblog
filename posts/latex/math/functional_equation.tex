\documentclass[11pt]{scrartcl}
\usepackage[sexy]{evan}





\begin{document}
\title{Functional Equations}
\maketitle

General Techniques:
\begin{itemize}
    \item Try values, get fixed point or at least known point
    \item Force cancellation
    \item Injectivity, Surjectivity, Bijectivity
    \item Exploit as many properties as possible then rewrite original equation.
    \item Injectivity at a point, injectivity of composition of function
    \item Cauchy FE
\end{itemize}

\section{IMO SL 2011}
\begin{Problem}[\url{https://artofproblemsolving.com/community/c6h488535p2737643}]
Determine all pairs $(f,g)$ of functions from the set of real numbers to itself that satisfy\[g(f(x+y)) = f(x) + (2x + y)g(y)\]for all real numbers $x$ and $y$.
\end{Problem}


\begin{Problem}[\url{https://artofproblemsolving.com/community/c6h488536p2737644}]
Determine all pairs $(f,g)$ of functions from the set of positive integers to itself that satisfy\[f^{g(n)+1}(n) + g^{f(n)}(n) = f(n+1) - g(n+1) + 1\]for every positive integer $n$. Here, $f^k(n)$ means $\underbrace{f(f(\ldots f)}_{k}(n) \ldots ))$.
\end{Problem}


\begin{Problem}[\url{https://artofproblemsolving.com/community/c6h418798p2363539}]
Let $f : \mathbb R \to \mathbb R$ be a real-valued function defined on the set of real numbers that satisfies
\[f(x + y) \leq yf(x) + f(f(x))\]
for all real numbers $x$ and $y$. Prove that $f(x) = 0$ for all $x \leq 0$.
\end{Problem}


\begin{Problem}[\url{https://artofproblemsolving.com/community/c6h488539p2737648}]
For any integer $d > 0,$ let $f(d)$ be the smallest possible integer that has exactly $d$ positive divisors (so for example we have $f(1)=1, f(5)=16,$ and $f(6)=12$). Prove that for every integer $k \geq 0$ the number $f\left(2^k\right)$ divides $f\left(2^{k+1}\right).$
\end{Problem}


\begin{Problem}[\url{https://artofproblemsolving.com/community/c6h418981p2365041}]
Let $f$ be a function from the set of integers to the set of positive integers. Suppose that, for any two integers $m$ and $n$, the difference $f(m) - f(n)$ is divisible by $f(m- n)$. Prove that, for all integers $m$ and $n$ with $f(m) \leq f(n)$, the number $f(n)$ is divisible by $f(m)$.
\end{Problem}



\section{IMO SL 2010}

\begin{Problem}[\url{https://artofproblemsolving.com/community/c6h356075p1935849}]
Find all function $f:\mathbb{R}\rightarrow\mathbb{R}$ such that for all $x,y\in\mathbb{R}$ the following equality holds\[
f(\left\lfloor x\right\rfloor y)=f(x)\left\lfloor f(y)\right\rfloor \]where $\left\lfloor a\right\rfloor $ is greatest integer not greater than $a.$
\end{Problem}


\begin{Problem}[\url{https://artofproblemsolving.com/community/c6h418682p2362286}]
Denote by $\mathbb{Q}^+$ the set of all positive rational numbers. Determine all functions $f : \mathbb{Q}^+ \mapsto \mathbb{Q}^+$ which satisfy the following equation for all $x, y \in \mathbb{Q}^+:$\[f\left( f(x)^2y \right) = x^3 f(xy).\]
\end{Problem}


\begin{Problem}[\url{https://artofproblemsolving.com/community/c6h418683p2362289}]
Suppose that $f$ and $g$ are two functions defined on the set of positive integers and taking positive integer values. Suppose also that the equations $f(g(n)) = f(n) + 1$ and $g(f(n)) = g(n) + 1$ hold for all positive integers. Prove that $f(n) = g(n)$ for all positive integer $n.$
\end{Problem}


\begin{Problem}[\url{https://artofproblemsolving.com/community/c6h356076p1935854}]
Find all functions $g:\mathbb{N}\rightarrow\mathbb{N}$ such that\[\left(g(m)+n\right)\left(g(n)+m\right)\]is a perfect square for all $m,n\in\mathbb{N}.$
\end{Problem}


\section{IMO SL 2009}

\begin{Problem}[\url{https://artofproblemsolving.com/community/c6h289055p1562848}]
Determine all functions $ f$ from the set of positive integers to the set of positive integers such that, for all positive integers $ a$ and $ b$, there exists a non-degenerate triangle with sides of lengths
\[ a, f(b) \text{ and } f(b + f(a) - 1).\]
(A triangle is non-degenerate if its vertices are not collinear.)
\end{Problem}

\begin{Problem}[\url{https://artofproblemsolving.com/community/c6h355779p1932913}]
Let $f$ be any function that maps the set of real numbers into the set of real numbers. Prove that there exist real numbers $x$ and $y$ such that\[f\left(x-f(y)\right)>yf(x)+x\]
\end{Problem}


\begin{Problem}[\url{https://artofproblemsolving.com/community/c6h355780p1932915}]
Find all functions $f$ from the set of real numbers into the set of real numbers which satisfy for all $x$, $y$ the identity\[ f\left(xf(x+y)\right) = f\left(yf(x)\right) +x^2\]
\end{Problem}


\begin{Problem}[\url{https://artofproblemsolving.com/community/c6h355797p1932942}]
Let $f$ be a non-constant function from the set of positive integers into the set of positive integer, such that $a-b$ divides $f(a)-f(b)$ for all distinct positive integers $a$, $b$. Prove that there exist infinitely many primes $p$ such that $p$ divides $f(c)$ for some positive integer $c$.
\end{Problem}


\section{IMO SL 2008}

\begin{Problem}[\url{https://artofproblemsolving.com/community/c6h215430p1191683}]
Find all functions $ f: (0, \infty) \mapsto (0, \infty)$ (so $ f$ is a function from the positive real numbers) such that
\[ \frac {\left( f(w) \right)^2 + \left( f(x) \right)^2}{f(y^2) + f(z^2) } = \frac {w^2 + x^2}{y^2 + z^2}
\]
for all positive real numbers $ w,x,y,z,$ satisfying $ wx = yz.$
\end{Problem}


\begin{Problem}[\url{https://artofproblemsolving.com/community/c6h287852p1555894}]
Let $ S\subseteq\mathbb{R}$ be a set of real numbers. We say that a pair $ (f, g)$ of functions from $ S$ into $ S$ is a Spanish Couple on $ S$, if they satisfy the following conditions:
(i) Both functions are strictly increasing, i.e. $ f(x) < f(y)$ and $ g(x) < g(y)$ for all $ x$, $ y\in S$ with $ x < y$;

(ii) The inequality $ f\left(g\left(g\left(x\right)\right)\right) < g\left(f\left(x\right)\right)$ holds for all $ x\in S$.

Decide whether there exists a Spanish Couple
on the set $ S = \mathbb{N}$ of positive integers;
on the set $ S = \{a - \frac {1}{b}: a, b\in\mathbb{N}\}$
\end{Problem}




\begin{Problem}[\url{https://artofproblemsolving.com/community/c6h287853p1555896}]
For an integer $ m$, denote by $ t(m)$ the unique number in $ \{1, 2, 3\}$ such that $ m + t(m)$ is a multiple of $ 3$. A function $ f: \mathbb{Z}\to\mathbb{Z}$ satisfies $ f( - 1) = 0$, $ f(0) = 1$, $ f(1) = - 1$ and $ f\left(2^{n} + m\right) = f\left(2^n - t(m)\right) - f(m)$ for all integers $ m$, $ n\ge 0$ with $ 2^n > m$. Prove that $ f(3p)\ge 0$ holds for all integers $ p\ge 0$.
\end{Problem}



\begin{Problem}[\url{https://artofproblemsolving.com/community/c6h287857p1555902}]
Let $ f: \mathbb{R}\to\mathbb{N}$ be a function which satisfies $ f\left(x + \dfrac{1}{f(y)}\right) = f\left(y + \dfrac{1}{f(x)}\right)$ for all $ x$, $ y\in\mathbb{R}$. Prove that there is a positive integer which is not a value of $ f$.
\end{Problem}



\begin{Problem}[\url{https://artofproblemsolving.com/community/c6h287873p1555934}]
For every $ n\in\mathbb{N}$ let $ d(n)$ denote the number of (positive) divisors of $ n$. Find all functions $ f: \mathbb{N}\to\mathbb{N}$ with the following properties:
$ d\left(f(x)\right) = x$ for all $ x\in\mathbb{N}$.
$ f(xy)$ divides $ (x - 1)y^{xy - 1}f(x)$ for all $ x$, $ y\in\mathbb{N}$.
\end{Problem}



\end{document}